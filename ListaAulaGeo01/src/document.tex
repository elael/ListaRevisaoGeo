%%This is a very basic article template.
%%There is just one section and two subsections.
%%This is a very basic article template.
%%There is just one section and two subsections.
\documentclass[a4paper,11pt,oneside,brazilian]{article}

\usepackage[utf8]{inputenx}
\usepackage[brazilian]{babel}
\usepackage{graphicx}
\usepackage{float}
\usepackage{textgreek}
\usepackage{mathtools}
\usepackage{enumerate}
\usepackage{xfrac}

\usepackage{pgf,tikz}
\usetikzlibrary{arrows}

\newcommand{\degre}{\ensuremath{^\circ}}
\newcommand{\bfig}{\begin{figure}[!h]\centering}
\newcommand{\efig}{\end{figure}}
\renewcommand{\thesection}{\Roman{section}} 

\definecolor{qqwuqq}{rgb}{0,0.39,0}
\definecolor{zzttqq}{rgb}{0.6,0.2,0}
\definecolor{xdxdff}{rgb}{0.49,0.49,1}
\definecolor{qqqqff}{rgb}{0,0,1}
\definecolor{cqcqcq}{rgb}{0.75,0.75,0.75}
\definecolor{uuuuuu}{rgb}{0.27,0.27,0.27}
\definecolor{uququq}{rgb}{0.25,0.25,0.25}
\definecolor{qqwuqq}{rgb}{0.0,0.39215686274509803,0.0}
\definecolor{ffqqqq}{rgb}{1.0,0.0,0.0}

\begin{document}

\title{Lista de Geometria Euclidiana}
\author{Prof. Eduardo Elael}
\date{Lista Revisão - PVNC 2014}
\maketitle


\begin{flushright}
  \emph{Data da realização:} 18/ago/2014
\end{flushright}

\section{Revisão}
\begin{enumerate}[{R}1]
 
  \item Ache a função \(y=f(x)\) que todos seus pontos \((x,y)\) tem a mesma
  distância para o ponto \(F(0,\sfrac{1}{4})\) e para a reta d,
  \(y=-\frac{1}{4}\).

  \nopagebreak[4]
  

		\begin{figure}[!h]
		\centering
			\begin{tikzpicture}[line cap=round,line join=round,>=triangle 45,x=1.0cm,y=1.0cm]
			\draw[->,color=black] (-4.021078886170084,0.0) -- (4.756065301642141,0.0);
			\foreach \x in {-4.0,-3.5,-3.0,-2.5,-2.0,-1.5,-1.0,-0.5,0.5,1.0,1.5,2.0,2.5,3.0,3.5,4.0,4.5}
			\draw[shift={(\x,0)},color=black] (0pt,2pt) -- (0pt,-2pt);
			\draw[->,color=black] (0.0,-2.5452374503323383) -- (0.0,2.511871353884455);
			\foreach \y in {-2.5,-2.0,-1.5,-1.0,-0.5,0.5,1.0,1.5,2.0,2.5}
			\draw[shift={(0,\y)},color=black] (2pt,0pt) -- (-2pt,0pt);
			\clip(-4.021078886170084,-2.5452374503323383) rectangle (4.756065301642141,2.511871353884455);
			\draw[color=qqwuqq,fill=qqwuqq,fill opacity=0.1] (2.5,-1.9159596726519414) -- (2.4159596726519417,-1.9159596726519414) -- (2.4159596726519417,-2.0) -- (2.5,-2.0) -- cycle; 
			\draw [color=ffqqqq,domain=-4.021078886170084:4.756065301642141] plot(\x,{(-4.0-0.0*\x)/2.0});
			\draw [dotted] (0.0,2.0)-- (2.5,0.78125);
			\draw [dotted] (2.5,-2.0)-- (2.5,0.78125);
			\draw (0.025796666910569758,-1.9509825967228445) node[anchor=north west] {$-\frac{1}{4}$};
			\draw (-0.4198944732965507,2.2325715726879927) node[anchor=north west] {$\frac{1}{4}$};
			\draw (2.4978968579260648,0.8539003123139668) node[anchor=north west] {$(x,y)$};
			\begin{scriptsize}
			\draw[color=ffqqqq] (-3.973538497881325,-1.903442208434085) node {$d$};
			\draw [fill=ffqqqq] (0.0,2.0) circle (1.5pt);
			\draw[color=ffqqqq] (0.20407312299341795,2.101835504893904) node {$F$};
			\draw [fill=black] (2.5,-2.0) circle (1.5pt);
			\draw [fill=ffqqqq] (2.5,0.78125) circle (2.0pt);
			\end{scriptsize}
			\end{tikzpicture}
		\end{figure}
  %\nopagebreak[4]
  \item Sabendo que os ângulos \(B\hat{C}D\) e \(E\hat{C}A\) são iguais. Mostre
  que
  \(\frac{\overline{AD}}{\overline{BD}}=\frac{\overline{AC}}{\overline{BC}}\).\\
  \textit{Dica: \textbf{semelhança} é vida!}

  \nopagebreak[4]
  

		\begin{figure}[!h]
		\centering
			\begin{tikzpicture}[scale=0.5,line cap=round,line join=round,>=triangle
			45,x=1.0cm,y=1.0cm] \clip(3.460230677202207,-4.100222850059955) rectangle (20.033436841484473,3.511300551255019);
			\draw [shift={(11.02,0.3)},color=qqwuqq,fill=qqwuqq,fill opacity=0.1] (0,0) -- (-73.46017599877106:0.5063097606196211) arc (-73.46017599877106:-22.364911495515354:0.5063097606196211) -- cycle;
			\draw [shift={(11.02,0.3)},color=qqwuqq,fill=qqwuqq,fill opacity=0.1] (0,0) -- (157.63508850448466:0.5063097606196211) arc (157.63508850448466:208.73035300774038:0.5063097606196211) -- cycle;
			\draw (5.0,-3.0)-- (11.02,0.3);
			\draw (11.02,0.3)-- (12.0,-3.0);
			\draw (5.0,-3.0)-- (19.0403378462527,-3.0);
			\draw (19.0403378462527,-3.0)-- (11.02,0.3);
			\draw [dotted] (11.02,0.3)-- (6.111598650550334,2.3195813148634206);
			\begin{scriptsize}
			\draw [fill=qqqqff] (5.0,-3.0) circle (1.5pt);
			\draw[color=qqqqff] (4.675374102689298,-3.1888652809446367) node {$A$};
			\draw [fill=qqqqff] (12.0,-3.0) circle (1.5pt);
			\draw[color=qqqqff] (11.99998863965315,-3.290127233068561) node {$B$};
			\draw [fill=qqqqff] (11.02,0.3) circle (1.5pt);
			\draw[color=qqqqff] (11.139262046599795,0.5409499556199071) node {$C$};
			\draw [fill=uuuuuu] (19.0403378462527,-3.0) circle (1.5pt);
			\draw[color=uuuuuu] (19.375234152678967,-3.2394962570065986) node {$D$};
			\draw [fill=xdxdff] (6.111598650550334,2.3195813148634206) circle (1.5pt);
			\draw[color=xdxdff] (6.228057368589469,2.549312006077739) node {$E$};
			\end{scriptsize}
			\end{tikzpicture}
		\end{figure}
		
  \pagebreak[4]
  \item Sendo \(\overline{CE}\) a altura relativa à base \(\overline{AB}\) e
  \(D\) o ponto médio de \(AC\). Calcule \(\overline{ED}\).

  \nopagebreak[4]
  

		\begin{figure}[!h]
		\centering
			\begin{tikzpicture}[line cap=round,line join=round,>=triangle 45,x=1.0cm,y=1.0cm]
			\clip(0.26,-5.16) rectangle (15.040000000000001,1.18);
			\draw[line width=0.4pt,color=qqwuqq,fill=qqwuqq,fill opacity=0.1] (6.212132034355964,-3.0) -- (6.212132034355964,-2.7878679656440357) -- (6.0,-2.7878679656440357) -- (6.0,-3.0) -- cycle; 
			\draw (2.0,-3.0)-- (13.0,-3.0);
			\draw (13.0,-3.0)-- (6.0,0.0);
			\draw (6.0,0.0)-- (2.0,-3.0);
			\draw (6.0,0.0)-- (6.0,-3.0);
			\draw (3.52,0.26) node[anchor=north west] {$\overline{AC} \, = \, 5$};
			\draw (9.8,-0.54) node[anchor=north west] {$\overline{CB} \, = \, 7.62$};
			\draw (6.12,-3.56) node[anchor=north west] {$\overline{BA} \, = \, 11$};
			\draw [dotted] (6.0,-3.0)-- (4.0,-1.5);
			\begin{scriptsize}
			\draw [fill=qqqqff] (2.0,-3.0) circle (1.5pt);
			\draw[color=qqqqff] (1.52,-3.2199999999999998) node {$A$};
			\draw [fill=qqqqff] (13.0,-3.0) circle (1.5pt);
			\draw[color=qqqqff] (13.14,-2.7199999999999998) node {$B$};
			\draw [fill=qqqqff] (6.0,0.0) circle (1.5pt);
			\draw[color=qqqqff] (6.14,0.28) node {$C$};
			\draw [fill=uuuuuu] (6.0,-3.0) circle (1.5pt);
			\draw[color=uuuuuu] (5.8,-3.32) node {$E$};
			\draw [fill=uuuuuu] (4.0,-1.5) circle (1.5pt);
			\draw [fill=uuuuuu] (9.5,-1.5) circle (1.5pt);
			\draw [fill=uuuuuu] (7.5,-3.0) circle (1.5pt);
			\draw [fill=uuuuuu] (4.0,-1.5) circle (1.5pt);
			\draw[color=uuuuuu] (3.56,-1.2000000000000002) node {$D$};
			\end{scriptsize}
			\end{tikzpicture}
		\end{figure}

  \item Archduke Franz Ferdinand decidiu fazer uma caixinha em formato de cubo
  para guardar sua bola de futebol. Ele não possuia a bola em mãos, mas como um
  bom austríaco estava com seu Livro de Regras 2013 / 2014 da CBF, onde se lia o
  trecho abaixo. Qual o volume mínimo que deve ter a caixinha de Franz para
  comportar a bola de futebol, certamente?

  \begin{quote}
  ''{\Large \textbf{Regra 2: A Bola}}\\\\
  \textbf{Características e medidas}\\
  \textbf{A bola:}
  \begin{itemize}
    \item será esférica
    \item será de couro ou qualquer outro material adequado
    \item terá uma circunferência não supe\-rior a 70 cm e não inferior a 68 cm
    \item terá um peso não superior a 450 g e não inferior a 410 g no começo da
    partida
    \item terá uma pressão equivalente a 0,6 – 1,1 atmosfe\-ras (600 – 1100
    g/cm$^2$) ao nível do mar (8.5 a 15.6 libras).''
  \end{itemize}
  \end{quote}
\end{enumerate}

\end{document}
